\documentclass[12pt, preprint]{aastex}
\usepackage{hyperref}
\usepackage{color}
\newcommand{\niceurl}[1]{\href{#1}{\textsl{#1}}}

\begin{document}

\title{A Field Guide to the Word ``Optimal'' in Astronomy}

\author{%
Dustin Lang\altaffilmark{1,2}%
}
\altaffiltext{1}{%
  Dunlap Institute and Department of Astronomy \& Astrophysics,
  University of Toronto,
  50 Saint George Street, Toronto, ON, M5S 3H4, Canada
}
\altaffiltext{2}{%
  Department of Physics \& Astronomy,
  University of Waterloo,
  200 University Avenue West, Waterloo, ON, N2L 3G1, Canada
}
\date{April 1, 2016}

\begin{abstract}
The word ``optimal'' is used widely in astronomy, but its semantics
may be unfamiliar to researchers from other fields.  Here we give
our interpretations of various ways the word is used in the literature.
\end{abstract}

\keywords{%
April 1; Optimality; Meta}

\section{Introduction}

Before exploring the use of the word in the astronomical literature, let
us present definitions that might be recognizable to researchers from other
disciplines.

\begin{description}
\item[Optimal (of a parameter):] achieving the maximum of a
  well-specified scalar objective function.  That is, a parameter
  $\hat{\theta}$ can be said to be \emph{optimal} in $f$ if and only
  if
  \[
  \hat{\theta} = \arg\max_{\theta} f(\theta)
  \]
\item[Optimal (of a method, algorithm, or procedure):] achieving the
  best possible value of a specified scalar performance metric among a
  specified group of algorithms.
  %
  % For example,  ... Cramer--Rao
  %
\item[Optimize:] the act of discovering an optimal value through analytic or
  numerical means.
\end{description}

\section{``Optimal'' in Astronomy}

A variety of uses of the word ``optimal'' can be found in astronomy.
Here we show a number of examples with our interpretations (XXX
translations?) of their semantics.
%
% Specific references to the literature have been suppressed to
% prevent embarassment, [but the authors of this article are coauthors
% on each quoted work].


\paragraph{``We present an optimal algorithm.''}
Translation: the authors present a good algorithm.  And by good, they
mean that they like it.  They like it because they wrote it, and it
even seemed to work better on the \emph{two} different test images
that they ran it on.  At least, it looked better to then.  They plan
to make the code public in the near future.


% We present a good algorithm.
% And by good, we mean that we like it.  We like it because we wrote it,
% and it even seems to work better on \emph{two} different test images
% that we ran it on.  At least, it looked better to us.  We plan to make
% the code public in the near future.

\paragraph{``\emph{A} is more optimal than \emph{B}.''}
Translation: Neither \emph{A} nor \emph{B} is any good.  The phrase
``more optimal'' of course indicates that the writer has never climbed
to the top of a hill and has not in any way quantitatively evaluated
\emph{A} versus \emph{B}.  Further, if \emph{A} were truly optimal, it
would be fruitless to search for a superior \emph{B}; the fact that
the claim is being made probably means that \emph{A} has not been
shown to be optimal.




\section{Conclusions}


\end{document}
